\chapter{Preparatory Work}

This section describes work conducted in preparation for this master thesis. The biggest portion of which went into researching previous singing voice synthesis methods. Some effort was put into acquiring a better understanding of the human speech production system. A good understanding of the speech production system is important for this thesis mainly for defining a synthesis model that closely mimics the speech production system. Some time was spend studying modern approaches for source filter separation, filter envelope estimation and glottal flow models. The separation of the vocal tract filtering and the vocal folds oscillation (source) is a reoccurring challenge in singing voice synthesis. Finally, some time was devoted to researching machine learning approaches to singing voice synthesis.

In preparation for this thesis, some of the methods proposed in previous papers, including filter envelope estimators (LPC, TE) and glottal flow models (LF, LF-Rd) have been implemented in Matalb in order to get familiar with their performance, strengths and weaknesses. A framework for extract pitch, overtone frequencies and phases of sung vowel excerpts have been implemented using the programming environment Matlab. It was used to explore the VocalSet dataset\cite{wilkins_vocalset:_2018} and to analyse the included singing voice experts regarding spectral content and temporal fluctuation of pitch and overtone amplitude and phase. 
