\usepackage[utf8]{inputenc}
%\usepackage[russian,german,ngerman]{babel}

% Titelei
\newcommand*{\Titel}{Efficient Singing Voice Synthesis}
\newcommand*{\PdfTitel}{Efficient Singing Voice Synthesis}
\newcommand*{\Untertitel}{William Rodewald}
% Header-Titel
\newcommand*{\HeaderTitel}{Master Thesis Proposal}

% Hauptautor
\newcommand*{\Autor}{William Rodewald}
\newcommand*{\AutorMatrikelNr}{372193}

% Manuell gesetztes Datum mit Ort
\newcommand*{\Datum}{Berlin, \today}

% Uni, Fakultät, Institut und Fachgebiet
\newcommand*{\Uni}{Technische Universität Berlin}
\newcommand*{\Fakultaet}{Fakultät I}
\newcommand*{\Institut}{Institut für Sprache und Kommunikation}
\newcommand*{\Fachgebiet}{Fachgebiet Audiokommunikation}

% Professor und Betreuer
\newcommand*{\Professor}{Prof. Dr. Stefan Weinzierl}
\newcommand*{\BetreuerA}{Dr. Henrik Hahn}
\newcommand*{\InstitutA}{Fachgebiet Audiokommunikation}
\newcommand*{\InstitutB}{Ableton AG}

% Weitere Autoren 
%\newcommand*{\AutorB}{William Rodewald}
%\newcommand*{\AutorBMatrikelNr}{372193}

%\newcommand*{\AutorC}{Maximilian Weber}
%\newcommand*{\AutorCMatrikelNr}{385153}


% Gruppenangabe oder nur "vorgelegt von:" schreiben.
\newcommand*{\Gruppe}{vorgelegt von:}

% Einstellen der Druckversion: Drucken oder PDF-Ansicht allein
\def\typeDrucken{1}  % Für Drucken (nicht ändern)
\def\typePDFonly{2}  % Für reine PDF-Verwendung (nicht ändern)
% Hier den passenden Schalter setzen! =\typeDrucken oder =\typePDFonly
\let\pdftype=\typePDFonly

% Definition, ob ein- oder zweiseitig gesetzt werden soll
\def\typeOnepage{1}  % Für einseigen Satz
\def\typeTwopage{2}  % Für zweiseitgen Satz
% Hier den passenden Schalter setzen! =\typeOnepage oder =\typeTwopage
\let\pagetype=\typeOnepage

% Bindekorrektur am inneren Rand
\newlength{\bindCorrection} 				%Deklaration, nicht ändern
\setlength{\bindCorrection}{0cm} 			%Zuweisung, kann geändert werden

% PDF-Ausgabe-Version über Minor-Nummer => 1.x
\pdfminorversion=5
\pdfcompresslevel=9