% Einbinden von Vektor- und Rastergraphiken
%\usepackage[pdftex]{graphicx}
%\usepackage[space]{grffile}

%\graphicspath{{"D:/fabian sicherung/Uni-Master/Masterarbeit/Dokument/Bilder/pdf/"}{"D:/fabian sicherung/Uni-Master/Masterarbeit/Dokument/Bilder/jpg/"}}
%Hier Pfad ändern für kleinere Bilder
%\graphicspath{{"D:/fabian sicherung/Uni-Master/Masterarbeit/Dokument/Bilder/pdf/"}{"D:/fabian sicherung/Uni-Master/Masterarbeit/Dokument/Bilder/jpg/komprimiert/"}}


% Beispiel einer Anwendung: 
%\begin{figure}[!htb] % <<<---- Am besten hier positionieren, sonst oben oder unten auf der Seite
%\centering   % <<<---- Zentrierte Darstellung
%\includegraphics[width=1.00\textwidth]{beispiel.pdf}  % <<<---- Fügt auf ganze Seitenbreite das Bild "beispiel.pdf" ein
%\caption{Beispielbild}  % <<<---- Bildunterschrift
%\label{fig:marke}  % <<<---- Marke zum Referenzieren
%\end{figure}

% Ermöglicht die Verwendung von eps-Graphiken
%\usepackage{epstopdf}
% Dieses Paket wandelt on-the-fly eps-Dateien in pdf-Dateien um, diese werden in dem Bilder-Ordner automatisch abgelegt!
% Benötigt \write18 enabled, d. h. --shell-escape für tetex, texlive oder --enable-write18 für miktex

%Einbinden von Tikz und pgf, so dass auch sehr große 3D-Grafiken dargestellt werden können
\usepackage{tikz,pgfplots}
\usetikzlibrary{spy} %Vergrößerung
%PGFPLot-Optionen in Reihenfolge: Kompatibilität, deutsche Zahlentrennzeichen, runde Graphecken, dickere Graphen (0.4 is standard), kleinere Zahlen an Achsen, Hintergrund leicht blau, Rahmen etwas dicker, Grid, NaNs nicht zeichnen, x-Achse nicht enlargen
\pgfplotsset{compat=1.5, /pgf/number format/use comma,  every axis plot post/.append style={line join=round}, every axis plot/.append style={line width=1 pt}, axis background/.style={fill=blue!3}, axis line style = thick, grid=major, unbounded coords=jump, enlarge x limits=false}%every axis plot post/.append style={mark=none}},/pgf/number format/fixed,

%Legendensachen: Strichstärke, vert. Abstand Zeilen, Schrift links ausrichten, minor ticks kürzer
\pgfplotsset{
legend image post style={line width=1pt},
legend style={row sep=2pt},
legend cell align=left,
minor tick length=0.1cm* 0.8,
minor x tick num=1,
minor y tick num=1
}

%Schriftgrößen
\pgfplotsset{
	tick label style={font=\footnotesize},
	label style={font=\small},
	legend style={font=\small}
}

%Color Cycle List für PGF
\pgfplotscreateplotcyclelist{mycolorlist}{%
	blue\\%
	red\\%
	green!60!black\\%
	cyan\\%
	orange\\%
	magenta\\%
	violet\\%
	lime\\%
	black\\%
}

\usetikzlibrary{pgfplots.external,pgfplots.colormaps,decorations.markings,chains,fit,shapes,decorations.pathreplacing,positioning}
\usepgfplotslibrary{patchplots,colormaps}

%Makro für colormap reversen
\makeatletter
\def\customrevertcolormap#1{%
	\pgfplotsarraycopy{pgfpl@cm@#1}\to{custom@COPY}%
	\c@pgf@counta=0
	\c@pgf@countb=\pgfplotsarraysizeof{custom@COPY}\relax
	\c@pgf@countd=\c@pgf@countb
	\advance\c@pgf@countd by-1 %
	\pgfutil@loop
	\ifnum\c@pgf@counta<\c@pgf@countb
	\pgfplotsarrayselect{\c@pgf@counta}\of{custom@COPY}\to\pgfplots@loc@TMPa
	\pgfplotsarrayletentry\c@pgf@countd\of{pgfpl@cm@#1}=\pgfplots@loc@TMPa
	\advance\c@pgf@counta by1 %
	\advance\c@pgf@countd by-1 %
	\pgfutil@repeat
	%\pgfplots@colormap@showdebuginfofor{#1}%
}%

\makeatother

%pfdpages für Titelblatt muss vor externalize und dort auch "ausgeschlossen" werden durch optimize
\usepackage{pdfpages}

%Funktionierende, Scheiss-Methode:
%\tikzexternalize[shell escape=-enable-write18, prefix=Bilder/externalized/, optimize command away=\includepdf]

%Intervall zeichnen
\usepackage{tikz-dimline}

%alle neu machen...
%\tikzset{external/force remake}