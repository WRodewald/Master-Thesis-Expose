% Dokumenteneinstellungen werden hier vorgenommen. Tiefere Layout-Einstellungen folgen separat.

% Hier u. a. auch möglich: 
% Spaltenanzahl (onecolumn, twocolumn)
% Kapitel nur auf rechten Seiten beginnen oder auf jeder (openright, openany)
% Beginn von Absätzen (parindent, halfparskip, parskip)

\documentclass[
headings=big,  						% Überschriftengröße (möglich: small, normal und big)
captions=tableheading,				% Tabellen-Captions als Überschriften formatieren
bibliography=totoc,			% Fügt das Literaturverzeichnis im Inhaltsverzeichnis hinzu
listof=totoc,						% Fügt die weiteren Verzeichnisse im Inhaltsverzeichnis hinzu
index=totoc,						% Fügt auch den Index im Inhaltsverzeichnis hinzu				
draft=false,							% Graphiken werden gesetzt. bei draft=true sind Graphiken nur als Rahmendummies verfügbar
numbers=noenddot				% keine Punkte nach kapitelnummern un so
]{scrreprt} 					  	% Wählen aus scrartcl, scrreprt und scrbook !!! Unten unbedingt den Schalter anpassen !!!
% Logische Schalter für Dokumentenklasse 
\def\typeScrartcl{1}  % Für scrartcl
\def\typeScrreprt{2}  % Für scrreprt
\def\typeScrbook{3}   % Für scrbook

% !! Schalter hier setzen, damit die Vorlage korrekt funktioniert !!
\let\doctype=\typeScrreprt