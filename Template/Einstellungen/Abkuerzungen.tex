%\usepackage[printonlyused]{acronym}
\usepackage{acronym}
%\Auch für Formelzeichen benutzt, dann nur ohne printonlyused

\newcommand{\acrounit}[1]{
	\acroextra{\makebox[18mm][l]{{#1}}}
}

%Abkürzungen in der Auflistung fett -> auskommentieren 
\renewcommand*{\aclabelfont}[1]{\normalfont{\normalsize{#1}}\hfill}


%Ausländische Abkrüzungen
\makeatletter
\newcommand{\acroforeign}[1]{}

% patch the environment to print the foreign definition:
\AtBeginEnvironment{acronym}{%
	\def\acroforeign#1{ (#1)}%
}

% patch the acronym definition to safe the foreign definition:
\patchcmd\AC@@acro
{\begingroup}
{\begingroup\def\acroforeign##1{\csdef{ac@#1@foreign}{##1, }}}
{}
{}

% renew the first output to include the foreign definition if given:
\renewcommand*{\@acf}[1]{%
	\ifAC@footnote
	\acsfont{\csname ac@#1@foreign\endcsname\AC@acs{#1}}%
	\footnote{\AC@placelabel{#1}\hskip\z@\AC@acl{#1}{}}%
	\else
	\acffont{%
		\AC@placelabel{#1}\hskip\z@\AC@acl{#1}%
		\nolinebreak[3] %
		\acfsfont{(\acsfont{\csname ac@#1@foreign\endcsname\AC@acs{#1}})}%
	}%
	\fi
	\ifAC@starred\else\AC@logged{#1}\fi
}
\makeatother
