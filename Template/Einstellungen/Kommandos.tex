% Hier benutzerdefinierte Kommandos einfügen. Einige vordefinierte Kommandos sind unten zu finden.

\newcommand{\e}[1]{\mathrm{e} #1}
\renewcommand{\i}[1]{\mathrm{i} #1}
% Erstellt den diag-Operator 
\DeclareMathOperator{\diag}{diag}
% Erstellt eine partielle Ableitung (oben und unten ein \partial)
\newcommand{\pfrac}[2]{\frac{\partial #1}{\partial #2}}
%Zweite partielle Ableitung
\newcommand{\pfractwo}[2]{\frac{\partial^2 #1}{\partial {#2}^2}}
% "Totale" Ableitung
\newcommand{\tdfrac}[2]{\frac{\mathrm{d} #1}{\mathrm{d} #2}}
% Differentiale am Integralende
\renewcommand{\d}{\,\mathrm{d}}
% "Partial"-Zeichen
\newcommand{\p}{\partial}
% Fette Symbole (auch griechische)
%\newcommand{\bs}[1]{\boldsymbol{#1}}
% Betrag
\newcommand{\abs}[1]{\left\lvert #1 \right\rvert}
% Norm
\newcommand{\norm}[1]{\left\lVert #1 \right\rVert}
% Divergenz
\DeclareMathOperator{\divv}{div}
% Gradient
\DeclareMathOperator{\grad}{grad}
% Rotation
\DeclareMathOperator{\rot}{rot}
% Cofaktor
\DeclareMathOperator{\cof}{Cof}
% Spur
\DeclareMathOperator{\Spur}{Sp}
% Sprungklammer
\newcommand{\jump}[1]{\left \llbracket #1 \right \rrbracket}
%Signum-Funktion
\newcommand{\sign}{\text{sign}}
% Physikalische Komponenten
\newcommand{\phy}[1]{{\langle #1 \rangle}}
% Inverse eines Symbols, mit angepasster virtueller Höhe,
% damit Indizes passen (z. B. \inverse{\sigma}^{ij} )
\newcommand{\inverse}[1]{\stackrel{-1}{#1}\!\!\vphantom{#1}}
% Repariert kaputte Einrückungen, z. B. nach aligns
\newcommand{\fixIndent}{ \\[-\parskip]}
% Abkürzungen zum Erstellen von Gleichungen
\newcommand{\beq}{\begin{equation}}
\newcommand{\eeq}{\end{equation}}
\newcommand{\beqq}{\begin{equation*}}
\newcommand{\eeqq}{\end{equation*}}
\newcommand{\bal}{\begin{align}}
\newcommand{\eal}{\end{align}}
% Formelzeichen mit untergesetzter Formel (oder Zeichen), mit Indexhöhenanpassung.
\newcommand{\sU}[2]{\underset{#2}{#1}\vphantom{#1}}
% Wie oben, ignoriert aber beim Satz die Breite der Formel
\newcommand{\sUs}[2]{\smash{\underset{#2}{#1}\vphantom{#1}}}

% Neue Referenzbefehle über varioref (kein fancyref, da dies keine Abkürzungen vorsieht)
\newcommand{\charef}[1]{Kap.\,\vref*{#1}}
\newcommand{\secref}[1]{Abschn.\,\vref*{#1}}
\newcommand{\ssecref}[1]{Unterabschn.\,\vref*{#1}}
\newcommand{\veqref}[1]{Gl.\,\eqref{#1}\vpageref{#1}}
\newcommand{\figref}[1]{Abb.\,\vref*{#1}}
\newcommand{\sfigref}[2]{Abb.\,\ref{#1}\subref{#2}\vpageref{#1}}
\newcommand{\tabref}[1]{Tab.\,\vref*{#1}}
\newcommand{\lstref}[1]{Lst.\,\vref*{#1}}
% Ohne Seitenangabe
\newcommand{\neqref}[1]{Gl.\,\eqref{#1}}
\newcommand{\ncharef}[1]{Kap.\,\ref*{#1}}
\newcommand{\nsecref}[1]{Abschn.\,\ref*{#1}}
\newcommand{\nssecref}[1]{Unterabschn.\,\ref*{#1}}
\newcommand{\nfigref}[1]{Abb.\,\ref*{#1}}
\newcommand{\nsfigref}[2]{Abb.\,\ref{#1}\subref{#2}}
\newcommand{\ntabref}[1]{Tab.\,\ref*{#1}}
\newcommand{\nlstref}[1]{Lst.\,\ref*{#1}}
% Anwendung z. B. mit: ...siehe \charef{cha:01}  -> ... siehe Kap. 3 auf S. 2
% Unterabbildungen z. B. mit: ... siehe \sfigref{fig:01}{sfig:01:a} -> ... siehe Abb. 3.2(a)

% Max: Quote Marks
\newcommand{\q}[1]{``#1''}

% URL Style
\urlstyle{same}

% remove bottom bar

\newcommand{\comment}[1]{{\color{red}{[#1]}}}


\usepackage{array}
\newcolumntype{?}{!{\vrule width 1pt}}