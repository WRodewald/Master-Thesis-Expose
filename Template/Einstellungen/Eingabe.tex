% Sichert die Copy&Paste-Fähigkeit
\usepackage{cmap}

% Korrekte Darstellung europäischer Zeichen wie Umlaute
\usepackage[OT2, T1]{fontenc}
% Beispiel: Ohne dieses Paket wäre ein ä eine Kombination von: a + . + . = drei Zeichen -> schlecht für Copy & Paste aus der PDF und für die Suche

% Einstellung des Eingabezeichensatz
\usepackage[utf8]{inputenc}
% Gewählt: UTF-8-Zeichensatz (für Windows: ansi nehmen)

% Umlaute bei Overleaf! (Max W)
%\usepackage[utf8x]{inputenc}
%\usepackage[german]

% Translation verschiedener Elemente des Dokumentes
%\usepackage[german,ngerman]{babel}
%\usepackage[ngerman]{babel}
%\addto{\captionsngerman}{%
%	\renewcommand{\bibname}{Quellenverzeichnis}
%}

% ENGLISH (M. Weber)
\usepackage[english]{babel}

% Z. B. die korrekte Übersetzung von Verzeichnisnamen. Mehrere Namen als Option sind möglich. Die zuletzt geladene Sprache ist standard, sollte ngerman (Deutsch mit Reformanpassung) sein.


% Ermöglicht Silbentrennung von zusammengesetzten Wörtern
\usepackage[shortcuts]{extdash}
% Z. B.: \hyphenation{Berta} \hyphenation{Ra-ke-te} Berta\-/Rakete
% Siehe Datei Silbentrennung