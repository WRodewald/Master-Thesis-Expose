% Neudefinition der Tabellen- und Abbildungsunterschriften
\addto\captionsngerman{
% Abb. statt Abbildung in Bildunterschriften
\renewcommand{\figurename}{Fig.}
% Tab. statt Tabelle
\renewcommand{\tablename}{Tab.}
}
% Ändern des Abstandes zwischen den Abkürzungen Abb. bzw. Tab. und der jeweiligen Nummer von ~ auf \,
\renewcommand*{\figureformat}{\figurename\,\thefigure\autodot}
\renewcommand*{\tableformat}{\tablename\,\thetable\autodot}

% Kein Einzug ab der zweiten Caption-Zeile
\setcapindent{0em}

% Abkürzungen Abb. und  Tab. mit Nummer in den Floats fett setzen
\setkomafont{captionlabel}{\bfseries}

% Bild- und Tabellenunterschriften verkleinern
\addtokomafont{caption}{\small}

% Vorschlag für besseres Float-Placement von http://mintaka.sdsu.edu/GF/bibliog/latex/floats.html
\renewcommand{\topfraction}{0.9}				% Max. Anteil von Floats oben
\renewcommand{\bottomfraction}{0.8}				% Max. Anteil von Floats unten
\setcounter{topnumber}{2}						% Zwei Floats oben möglich
\setcounter{bottomnumber}{2}					% Zwei Floats unten möglich
\setcounter{totalnumber}{4}     				% Max. Anzahl Flaots auf einer Seite
\setcounter{dbltopnumber}{2}    				% Wie topnumber für zwei Spalten
\renewcommand{\dbltopfraction}{0.9}				% Wie topfraction für zwei Spalten
\renewcommand{\textfraction}{0.07}				% Min. text mit Floats
\renewcommand{\floatpagefraction}{0.7}			% Floatanteil, ab dem es Extraseiten gibt
\renewcommand{\dblfloatpagefraction}{0.7}		% Analog zu floatpagefraction für zwei Spalten
