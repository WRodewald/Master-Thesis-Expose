% Glossar laden - mit Abkürzungsverzeichnis 
\usepackage[acronym, toc]{glossaries}
% Später aufrufen: makeindex -s Dokument.ist -t Dokument.glg -o Dokument.gls Dokument.glo

% Symbolverzeichnis einstellen
\newglossary{symbols}{sym}{sbl}{Symbolverzeichnis}

% Üebrsetzen von Glossar und Abkürzungsverzeichnis
%\deftranslation[to=ngerman]{Acronyms}{Abkürzungsverzeichnis}
%\deftranslation[to=ngerman]{Glossary}{Glossar}

% Entfernen von Punkten hinter den Verzeichniseinträgen
\renewcommand*{\glspostdescription}{}

% Glossar festsetzen
\makeglossaries
% Als standard kein Kommentar bei Symbolen (später überschreibbar)
\newcommand{\symbComm}{}

% Style für Symbolverzeichnis
\newglossarystyle{symb3spaltig}{
% Umgebung: longtable 
\renewenvironment{theglossary}
{\symbComm{} \begin{longtable}{@{}p{1.5cm}p{1.5cm}p{\dimexpr\linewidth-1.5cm-1.5cm-4\tabcolsep\relax}@{}}} 
{\end{longtable}}
% Tabellenkopf 
\renewcommand*{\glossaryheader}{ 
\toprule 
\textbf{Symbol} & \textbf{Einheit} & \textbf{Beschreibung} \\ %Beschriftung 
\midrule 
\endhead}% 
% keine Überschriften zwischen Gruppen 
\renewcommand*{\glsgroupheading}[1]{}
% Haupteinträge in einer Zeile: 
\renewcommand*{\glossaryentryfield}[3]{
\glsentryuseri{##1}% Symbol 
& \glsentryuserii{##1}% Einheit 
& ##3% Beschreibung 
\\% Zeilenende 
}
% nichts zwischen Gruppen 
\renewcommand*{\glsgroupskip}{}% 
} 

% Ausgabe:
% Glossar: 			\printglossary[style=index]
% Abkürzungen: 	\printglossary[type=\acronymtype,style=list]
% Symbole:			\printglossary[type=symbols,style=list]

% Beispiele für die Verwendung

% Für einen Glossareintrag:
% \newglossaryentry{wagen}{name=Wa-gen,description={Macht Brum Brum},plural=Wa-gen,sort=wagen} 
% Dereferenzierung: \gls{wagen} (klein) \Gls{wagen} (fängt groß an) \GLS{wagen} (Kapitalschrift) \glspl{wagen} (Plural) ...\Glspl. \GLSpl - Referenz ohne Text: \glsadd
% Anderes Bsp.: \newglossaryentry{ohm}{name=ohm,symbol={\ensuremath{\Omega}},description=unit of electrical resistance}

% Für eine Abkürzung: 
% \newacronym{led}{LED}{light-emitting diode}

% Für Symbolverzeichniseintrag:
% \newglossaryentry{pi}{type=symbols,name={\ensuremath{\pi}},sort=pi, description={ratio of circumference of circle to its diameter}}