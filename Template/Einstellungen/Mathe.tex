% Wichtige Befehle zum schönen Mathesatz
\usepackage{amsmath}
% Umgebungsbeispiele: \equation, \align, \multline, \split, \boxed
% Kontrollkommandos: \subequations, \displaybreak, \intertext, \left, \right, \text
% Matrizen: matrix, pmatrix (), bmatrix [], Bmatrix {}, vmatrix ||, Vmatrix ||||
% u. v. m.

% Schriftarten zum Mathesatz
\usepackage{amsfonts}
% Z. B. \mathbb{R} für die reellen Zahlen

% Riesige Liste von mathematischen Symbolen und Operatoren
\usepackage{amssymb}
% Z. B. \in, \leq, \geq, \not, \sim, \approx, ...

% Weitere Symbole
\usepackage{stmaryrd}
% Z. B. \llbracket .. \rrbracket

% Unterdrückt unlogische Warnungen von stmaryrd
\SetSymbolFont{stmry}{bold}{U}{stmry}{m}{n}

% Für Theoreme
\usepackage{amsthm}
%\newtheorem{lemma}{Lemma} 
%\begin{lemma}
%...
%\end{lemma}

% Eine weitere Liste mit Symbolen, die häßlichere ablösen
\usepackage{latexsym}
% Z. B. \leadsto

% Normgerechtes Setzen von Zahlen mit SI-Einheiten
\usepackage[locale=DE]{siunitx}

%\usepackage[%per=slash,
%decimalsymbol=comma,
%loctolang={DE:ngerman,UK:english},
%]{siunitx}

\sisetup{exponent-product=\cdot,load-configurations=binary}
% \unit[1]{kg} = \SI{1}{\kilo\gram} == \SI{1}{kg} -> $1\,\text{kg}$
% \num{2.3d-2} -> $2,3 \cdot 10^{-2}$
% Stellt ebenfalls neuen Spaltentyp für Zahlen in Tabellen bereit für Ausrichtung am Dezimalzeichen:
% S[key=value, key=value, ...]
% Möglich sind hier z. B.:
% table-format = +3.2e+4 für 2 Vorkomma-, 4 Nachkomma- und 3 Exponentialstellen
% table-number-alignment=center oder right (Wo soll das Komamzeichen ausgerichtet werden?)
% table-figures-exponent=true oder false (Sollen die Exponenten auch ausgerichtet werden?)
% Weiterer neuer Spaltentyp s[key=value, ...] für Einheiten
% Wichtig hier: Kopfzeilen mit Text in {Text} versehen, bei mehreren Wörtern mit \multicolumn{1}{c}{Text} arbeiten. 

% Ermöglicht das Durchstreichen von Termen
\usepackage{cancel}
% \cancel{4}

% Verschiedene Möglichkeiten, Text zu unterstreichen. Wird in dieser Mathe-Datei eingefügt, da dieses Paket primär für Matrizen und Vektoren genutzt wird
\usepackage[normalem]{ulem}
% Z. B. für Vektoren \uline{v} oder für Matrizen \uuline{m}
% Aber auch für Text: \uwave{wellig}, \sout{falsch}, \xout{entfernt}, \dashuline{- - - }, \dotuline{. . . }

% Repariert die kaputte Kommasetzung für deutsche Kommazahlen im Mathemodus. Ursprünglich setzt Tex nach jedem Komma ein kleines Leerzeichen, aus 3,7 wird daher 3,\,7 = falsch.
\usepackage{icomma}
% Jetzt geht das Schreiben von 3,7 ohne Tricks. Ab jetzt nach Kommata mit -- gewünschten -- Abständen in Formeln ein Leerzeichen setzen, LaTex setzt dann ein kleines \,

% Setzen von Einheiten
\usepackage{units}
% Ermöglicht \unit[Zahl]{Einheit} und \unitfrac[Zahl]{Zähler}{Nenner} sowie nur \nicefrac{a}{b}
% Nicefrac setzen im Text statt a/b mit einem einfachen Slash die Terme abgesetzt.

% Darstellung von gerade gesetzten griechischen Buchstaben
\usepackage{upgreek}
% Verwendung z. B. mit: \uppi für ein gerades \pi

% Tensor-Notation / Indizes mit Abständen
\usepackage{tensor}
% Verwendung: M \indices{^a_b^{cd}_e}

% Erweiterter Blackboard-font
\usepackage{bbm}
% Verwendung: \mathbbm{N} - geht auch mit \boldsymbol

% Neue und überarbeitete Integralzeichen, ebenfalls bessere Abstände bei Mehrfachintegralen
\usepackage{esint}
% Verwendung: Z. B.: \oint

% Wurzel abschließen
\usepackage{letltxmacro}
\makeatletter
\let\oldr@@t\r@@t
\def\r@@t#1#2{%
\setbox0=\hbox{$\oldr@@t#1{#2\,}$}\dimen0=\ht0
\advance\dimen0-0.2\ht0
\setbox2=\hbox{\vrule height\ht0 depth -\dimen0}%
{\box0\lower0.4pt\box2}}
\LetLtxMacro{\oldsqrt}{\sqrt}
\renewcommand*{\sqrt}[2][\ ]{\oldsqrt[#1]{#2}}
\makeatother

% Gleichungsnummerierung
\ifcase\doctype 
\or \numberwithin{equation}{section}
\or 
\or  
\fi 

%Gleichstrom
\newcommand{\mathdirectcurrent}{\mathrel{\mathpalette\mathdirectcurrentinner\relax}}
\newcommand{\mathdirectcurrentinner}[2]{%
	\settowidth{\dimen0}{$#1=$}%
	\vbox to .85ex {\offinterlineskip
		\hbox to \dimen0{\hss\leaders\hrule\hskip.85\dimen0\hss}
		\vskip.35ex
		\hbox to \dimen0{\hss
			\leaders\hrule\hskip.17\dimen0
			\hskip.17\dimen0
			\leaders\hrule\hskip.17\dimen0
			\hskip.17\dimen0
			\leaders\hrule\hskip.17\dimen0
			\hss}
		\vfill
	}%
}
\newcommand{\textdirectcurrent}{\mathdirectcurrentinner{\textstyle}{}} 