% Paket für schnellen Beispieltext zum Füllen
%\usepackage{blindtext}
% Aus der Hilfe:
% \blindtext creates some text,
% \Blindtext creates more text.
% \blinddocument creates a small document with sections, lists...
% \Blinddocument creates a large document with sections, lists...

% Gant Diagram  (MAX)
\usepackage{pgfgantt}

% Linebreak in Table
\usepackage{makecell}

% Tabellen
\usepackage{tabularx,ragged2e}
%\newcolumntype{C}{>{\Centering\arraybackslash}X}

% Wichtige Macro-Definitionen 
\usepackage{etoolbox}
% Z. B. wichtig für logische Abfragen:
% \ifstrequal{String}{Vergleichstring}{...passiert when gleich}{...ansonsten}
% Anderes Beispiel, s. "Anpassen der Tiefe von nummerierten Überschriften" in Satzeinstellungen

% Weitertes Macro-Paket. Zwar depreciated, dennoch benötigt, da es in Skripten Kommandos auflöst
\usepackage{ifthen}
% Bsp: \ifthenelse{\equal{\kommando1}{\kommando2}}{true}{false}

% Ermöglicht das Verwenden von Listen mit kleineren vertikalen Abständen
\usepackage{mdwlist}
% Statt itemize einfach itemize* verwenden. Analog dazu existiert enumerate* und description*
% Diese Umbegungen erlauben auch Unterbrechungen, z. B.  mit \suspend{itemize*} bla bla \resume{itemize*}

%Bei Itemize davor und danach größere Abstände
\usepackage{enumitem}
\setlist[itemize]{topsep=5pt}

%Das gleiche für Quotes
\usepackage{quoting}
\quotingsetup{vskip=5pt}

% Ermöglicht erweiterte Referenzen. Soweit angepasst, dass schöne Makros möglich sind (siehe Kommando-Definitionen)
\usepackage[english]{varioref}
\renewcommand*{\reftextfaraway}[1]{auf S.\,\pageref{#1}}%
\renewcommand*{\reftextcurrent}{\unskip}%
% generell neu: \vpageref, \vref und \vref*, diese Referenzierungen geben die Seite mit an

% Ermöglichung von Unterabbildungen
\usepackage{subfig}
% Verwendung in figure: \subfloat[Optionaler Verzeichnistext][Unterschrift \label{sfig:bla}]{...Zeug...}

% Das normgerechte Eurozeichen
\usepackage{eurosym}
% Verwendung mit \euro

% Kommandos nach Ende einer Seite absetzen
\usepackage{afterpage} 
% Wichtigste Anwendung ist das Flushen von Floats: \afterpage{\clearpage}

 % Nassi-Schneidermann Unterstützung
%\usepackage{Template/Pakete/nassi}
%\setiftext{W}{F}
%\nassiwidth=\textwidth
% Wichtige Befehle:
% \NSD{}
% \WHILE{Text}{} \ENDWHILE
% \ACTION{Text}

% Einrahmen von Text
%\usepackage{framed}
%\addtolength{\FrameRule}{1pt} % dickeren Rahmen
% Verwendung: \begin{framed} ... \end{framed}

%Anhang
\usepackage[title]{appendix}

%Runden
\usepackage{xparse}

\ExplSyntaxOn
\DeclareExpandableDocumentCommand \ceil { O{0} m }
{ \fp_eval:n { ceil(#2,#1) } }
\ExplSyntaxOff

\ExplSyntaxOn
\DeclareExpandableDocumentCommand \floor { O{0} m }
{ \fp_eval:n { floor(#2,#1) } }
\ExplSyntaxOff

% Table Caption Distance
\usepackage{caption}
\captionsetup[table]{skip=10pt}

% Max Weber
\usepackage[nodayofweek]{datetime}
\usepackage{layouts}
\usepackage{mathrsfs}


\usepackage{floatrow}

% nicer inverse
\newcommand{\inv}{^{\raisebox{.2ex}{$\scriptscriptstyle-1$}}}