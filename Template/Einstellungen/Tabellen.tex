% Tabellen mit fester Gesamtbreite
\usepackage{tabularx}
% Beispiel: \begin{tabularx}{6cm}{lrX} <- X ist die variable Größe

% Mehrseitige, lange Tabellen
\usepackage{longtable}
% Wichtig: Setzen von \endfirsthead, \endhead, \endfoot, \endlastfoot am Anfang nach \begin{longtable}

% Ermöglichung von gedrehten Tabellen
\usepackage{rotating}
% gedrehte Tabelle mit der neuen Float-Tabellenumgebung \begin{sidewaystable} erstellen

% Professionelles Tabellenlayout
\usepackage{booktabs}
% Ermöglicht die Verwendung von \toprule, \midrule, \bottomrule, \cmidrule
% Wichtig für wissenschaftlich und ästhetisch ansprechende Tabellen


%Columntype, den ich haben will
\newcolumntype{M}[1]{>{\centering\arraybackslash}m{#1}}

\usepackage{array}
\newcolumntype{L}[1]{>{\raggedright\let\newline\\\arraybackslash\hspace{0pt}}m{#1}}
\newcolumntype{C}[1]{>{\centering\let\newline\\\arraybackslash\hspace{0pt}}m{#1}}
\newcolumntype{R}[1]{>{\raggedleft\let\newline\\\arraybackslash\hspace{0pt}}m{#1}}


%Damit die scheiss Tabellen nicht mehr rumspringen
\usepackage{float}